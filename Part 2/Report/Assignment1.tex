\chapter{Exercise 1}
general model for the truck and villages:
Initialization:
\begin{align*}
T_{0,location} =& S \\
A_{0,stock} =& 40 \\
B_{0,stock} =& 30 \\
C_{0,stock} =& 145 \\
\end{align*}

Locations changes and stock changes
\begin{align*}
\forall_{t \in \mathbb{N}}T_{t,location}=S& \implies \\
				  (&T_{t+1,load} = 300 \\
		   \land  &( \\
			   &\; (T_{t+1,location}=A \land A_{t+1,stock} = A_{t,stock}-29 \\
			   &\; \land B_{t+1,stock} = B_{t,stock}-29 \land C_{t+1,stock} = C_{t,stock}-29) \\
			   &\oplus \\
			   &\; (T_{t+1,location}=B \land A_{t+1,stock} = A_{t,stock}-21 \\
			   &\; \land B_{t+1,stock} = B_{t,stock}-21 \land C_{t+1,stock} = C_{t,stock}-21) \\
			   &)) \\
\forall_{t \in \mathbb{N}}T_{t,location}=A \implies \\
				  (&\exists_{0 \le u \le T_{t,load},u+A_{t,stock}<120}( T_{t+1,load} = T_{t,load}-u \\
		   \land  &( \\
		   &\; (T_{t+1,location}=S \land A_{t+1,stock} = A_{t,stock}+u-29 \\
		   &\; \land B_{t+1,stock} = B_{t,stock}-29 \land C_{t+1,stock} = C_{t,stock}-29) \\
		   &\oplus \\
		   &\;(T_{t+1,location}=B \land A_{t+1,stock} = A_{t,stock}+u-17 \\
		   &\; \land B_{t+1,stock} = B_{t,stock}-17 \land C_{t+1,stock} = C_{t,stock}-17) \\
           &\oplus \\
           &\; (T_{t+1,location}=C \land A_{t+1,stock} = A_{t,stock}+u-32 \\
           &\; \land B_{t+1,stock} = B_{t,stock}-32 \land C_{t+1,stock} = C_{t,stock}-32) \\
           &)))
\end{align*}


\begin{align*}
 \forall_{t \in \mathbb{N}}T_{t,location}=B \implies& \\
 				  (&\exists_{0 \le u \le T_{t,load}u+B_{t,stock}<120}( T_{t+1,load} = T_{t,load}-u \\
		   \land  &( \\
		   &\; (T_{t+1,location}=S \land A_{t+1,stock} = A_{t,stock}-21 \\
		   &\;\land B_{t+1,stock} = B_{t,stock}+u-21 \land C_{t+1,stock} = C_{t,stock}-21) \\
		   &\oplus \\
		   &\;(T_{t+1,location}=A \land A_{t+1,stock} = A_{t,stock}-17 \\
		   &\;\land B_{t+1,stock} = B_{t,stock}+u-17 \land C_{t+1,stock} = C_{t,stock}-17) \\
           &\oplus \\
           &\;(T_{t+1,location}=C \land A_{t+1,stock} = A_{t,stock}-37 \\
           &\;\land B_{t+1,stock} = B_{t,stock}+u-37 \land C_{t+1,stock} = C_{t,stock}-37) \\
           &))) \\
 \forall_{t \in \mathbb{N}}T_{t,location}=C \implies& \\
 				  (&\exists_{0 \le u \le T_{t,load}u+C_{t,stock}<200}( T_{t+1,load} = T_{t,load}-u \\
		   \land  &( \\
		   &\; (T_{t+1,location}=A \land A_{t+1,stock} = A_{t,stock}-32 \\
		   &\; \land B_{t+1,stock} = B_{t,stock}-32 \land C_{t+1,stock} = C_{t,stock}+u-32) \\
		   &\oplus \\
		   &\; (T_{t+1,location}=B \land A_{t+1,stock} = A_{t,stock}-37 \\
		   &\; \land B_{t+1,stock} = B_{t,stock}-37 \land C_{t+1,stock} = C_{t,stock}+u-37) \\
		   &))) \\
\end{align*}

Requirements to the stock:
\begin{align*}
\forall_{t\in \mathbb{N}}A_{t,stock} >= 0 \\
\forall_{t\in \mathbb{N}}B_{t,stock} >= 0 \\
\forall_{t\in \mathbb{N}}C_{t,stock} >= 0 \\ 
\end{align*}

Notes:
\begin{itemize}
\item Assumption, the truck is always moving, and does not stand still in any place \\
\item Assumption, the truck is always fully loaded at S
\end{itemize}

\section{sub question A}
To check this, the food supply may never run out and if it does, the truck should not be present. Whe have checked this with a limited number of timesteps in SMT. We checked first with a script with 1 timestep and then added a timestep until we reached a unsat. We needed 22 to timesteps for that.


\section{sub question B}
If the state is recuring, than it is possible to create an infinite loop where the requirement from question A holds. To elaborate a state $t_1 \in \mathbb{N}$, should be the same state as $t_2 \in \mathbb{N}$, where $t_1$ happened earlier than $t_2$, e.g. $t_1 < t_2$. Two states are the same iff $T_{t_1,location} = T_{t_1,location}$, $A_{t_1,stock} = A_{t_2,stock}$, $B_{t_1,stock} = B_{t_2,stock}$ and $C_{t_1,stock} = C_{t_2,stock}$.

This is also checked with the same model as question A, but a if a state is recuring than an loop can be created and all the villages are fed. So the describes requirement is added, that at least 2 states should be the same. With this we started with 2 timesteps and increased it, until we found a solution. At 12 timesteps we found a solution.In table \ref{table:LoopVillages}, the steps are shown. The highlighted steps (step 5 and 12) are equal in state, that way we can loop from 5 to 12 and feed the villages.

\begin{table}[]
	\centering
	\caption{Loop villages}
	\label{table:LoopVillages}
	\begin{tabular}{@{}llllll@{}}
		\toprule
		$t$ & $T_{t,Load}$ & $T_{t,location}$ & $A_{t,stock}$ & $B_{t,stock}$ & $C_{t,stock}$ \\ \midrule
		0        & 0     & S    & 40     & 30     & 145    \\
		1        & 320   & B    & 19     & 9      & 124    \\
		2        & 281   & A    & 2      & 31     & 107    \\
		3        & 164   & B    & 102    & 14     & 90     \\
		4        & 59    & S    & 81     & 98     & 69     \\
		\rowcolor[HTML]{FE996B} 
		5        & 320   & A    & 52     & 69     & 40     \\
		6        & 253   & C    & 87     & 37     & 8      \\
		7        & 67    & B    & 50     & 0      & 157    \\
		8        & 0     & S    & 29     & 46     & 136    \\
		9        & 320   & A    & 0      & 17     & 107    \\
		10       & 201   & B    & 102    & 0      & 90     \\
		11       & 82    & S    & 81     & 98     & 69     \\
		\rowcolor[HTML]{FE996B} 
		12       & 320   & A    & 52     & 69     & 40     \\ \bottomrule
	\end{tabular}
\end{table}

\section{sub question C}
For the last sub question, we have run both solutions, but with a maximal $T_{load}$ of 318. The only solution has finished was that of A. It gave an unsat with 68 timesteps. That means that 318 is too little to feed the villages 
